\documentclass{article}
\usepackage{amsmath}
\usepackage{amsfonts}
\usepackage{graphicx}
\usepackage{ctex,hyperref}



\title{二叉搜索树删除函数的实现与测试}
\author{}
\date{}

\begin{document}

\maketitle



\section{remove函数的实现}
构建 detachMin 函数从给定节点 \( t \) 开始寻找最小节点。最小节点必定没有左子树(否则在左子树存在更小的节点)。将指针指向最小节点的右孩子,并保存并返回该最小节点。在 remove 函数中,如果要删除的目标节点有两个孩子,就可以使用 detachMin 找到其右子树中的最小节点(并让其父亲和最小节点的右孩子相连),用该最小节点替换目标节点,以确保仍然是一个有效的二叉搜索树。

\section{测试输出}
用以下值初始化二叉搜索树: 

1,
3,
5,
7,
10,
12,
15,
18,
19,
20。

测试场景包括按如下方式删除节点:

- 删除叶子节点 20。
- 删除只有右孩子的节点:18。
- 删除只有左孩子的节点:3。
- 删除有两个孩子的节点:15。
- 尝试删除不存在的元素:100。

实际输出结果如下:

初始树:
1
3
5
7
10
12
15
18
19
20

删除 20 后:
1
3
5
7
10
12
15
18
19

删除 18 后:
1
3
5
7
10
12
15
19

删除 3 后:
1 
5 
7 
10 
12 
15 
19

删除 15 后:
1 
5 
7 
10 
12 
19 

尝试删除不存在元素100后:
1 
5 
7 
10 
12 
19 

结果与我们的预期一致,程序正常运行。

\end{document}