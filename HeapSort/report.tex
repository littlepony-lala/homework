\documentclass{article}
\usepackage{amsmath}
\usepackage{amsfonts}
\usepackage{graphicx}
\usepackage{ctex,hyperref}


\title{HeapSort报告}
\author{}
\date{}


\begin{document}

\maketitle
\section{Heapsort.h的设计思路}
将所给数组使用\texttt{std::sort\_make}构建堆后,遍历每一个节点使用\texttt{std::pop\_heap}将最大堆的根移到当前堆的最后,并使堆的范围减小1,最终形成升序的序列。
\section{测试流程}
check函数用于判断函数是否成功将数组成功排序,\texttt{test}函数用于记录排序的时间、判断排序是否成功;在main函数中分别对\texttt{main}函数中分别对\texttt{Heapsort.h} 和\texttt{std::heap\_sort}运行\texttt{test}检查各类数组是否都能正确排序并且比较运行时间。
\section{测试结果}
\begin{table}[h!]
    \centering
    \begin{tabular}{|l|c|c|}
    \hline
    序列类型 & My Heapsort 时间 & Std::heap\_sort 时间 \\
    \hline  
    随机序列 & 87ms & 43ms \\
    \hline 
    有序序列 & 34ms & 19ms \\
    \hline
    逆序序列 & 40ms & 37ms \\
    \hline
    部分重复序列 & 39ms & 33ms \\
    \hline
    \end{tabular}
    \caption{测试结果对比表}
    \end{table}
\end{document}
